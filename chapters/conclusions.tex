\chapter*{Conclusions and Future Work} 
\addcontentsline{toc}{chapter}{Conclusions and Future Work}

Clustering comparison metrics are important for benchmarking and evaluating the output of a clustering method when a ground truth is provided. These scores can also be used to compare different clustering methods or different configurations i.e. different values for the parameters of the same algorithm. Element-Centric Similarity (ECS) is a score that is not biased on the size, shape or the number of clusters and which provides information about the global and local agreement between two partitions. The \verb|ClustAssess| package provides the first R implementation of this score which also performs the calculation in an scalable, both space and time optimized manner.

We also emphasized the importance of the parameter configuration on the clustering output, but also the undesired impact that factors such as random seed can have on the final results. The solution we proposed in \verb|ClustAssess| was to provide a stability pipeline, which follows the steps from the PhenoGraph algorithm. The purpose of the pipeline is to provide visual assessments of the stability and robustness of different configuration at the change of random seed and to assist the user into choosing a configuration that leads to consistent, not seed reliant results.

As future work we are looking to furtherly optimize the calculation of the ECS score based on the final form as it was presented in Remark \ref{remark:ecs-constant}. At the moment only the case of disjoint partitions was optimized, so the calculation of the ECS score between overlapping or hierarhical partitions is also something to be improved. 

The introduction of the ECS threshold term and of the evaluation of close similarity between partitions could lead to further improvements in the time execution. The issue is that a specific value of the ECS value changes its interpretation when the size of the datasets changes. A ECS value of 0.99 could be interpreted as only a few points labeled differently for small datasets or entire region differences for larger ones. We intend in finding a correspondence between the ECS value and the possible number of points that changed their cluster label.

Finally, we are constantly planning on improving the quality and the usefuleness of the plots involved in the stability pipeline.
