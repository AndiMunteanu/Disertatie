\chapter{Description of methods}

This chapter contains informations about the methods used for graph clustering, the sequencing and processing the biological data and eventualy mentions of other works / papers that were focusing on assessing the robustness on changing the seed.

\section{Graph Clustering}

\subsection{Short intro about what graph clustering is}

\subsection{Why graph clustering instead other traditional methods such as k-means, density based techniques etc}

\subsection{Types of graph clustering}

\subsection{Community detection - Optimizing the quality function}

\subsection{Louvain}

\subsection{Louvain refined}
\subsection{SLM}
\subsection{Leiden}

\section{PhenoGraph pipeline}
    \subsection{Describing the pipeline}
    Present the steps that describe the pipeline. (Dimensionality reduction, graph building and graph clustering)
    \subsubsection{About dimensionality reduction}
    
    \subsection{How to convert matrix data into a graph using kNN}
    \subsection{SNN - providing weights using Jaccard Similarity Index}
    

\section{Element-Centric Similarity}
    \subsection{Description about how it works}
    Describe the intuition behind ECS: the idea of the bipartite graph between points and clusters.

    More details about how to calculate ECS. Talk about the affinity matrix and the L1 distance.
    \subsection{Properties, comparison with other clustering metrics}
    Present some limitation of other clustering metrics such as bias toward cluster sizes, shapes and so on. Perhaps present some comparison figures from the main article.

    Present some properties of ECS:
    \begin{enumerate}
        \item the fact that it can be used not only for flat disjoint clusterings, but also for overlapping or hierarchical partitions
        \item it overcomes the biases present in the other clustering metrics
        \item ECS illustrates the overall similarity between two partitions but also can help in identifying the points where the clustering are not similar
    \end{enumerate}
    \subsection{ECC}
    Talk about how ECC is calculated

\section{Intro info about biological data and sequencing techniques}
Tell about sequencing techniques, how the initial data looks, about cells, genes, what they mean, what is the role and the purpose of the clusters in the biological interpretation.