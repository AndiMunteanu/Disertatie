\chapter{Description of methods}

This chapter contains informations about the methods used for graph clustering, the sequencing and processing the biological data and eventualy mentions of other works / papers that were focusing on assessing the robustness on changing the seed.

\section{Graph Clustering}

\subsection{Short intro about what graph clustering is}

\subsection{Why graph clustering instead other traditional methods such as k-means, density based techniques etc}

\subsection{Types of graph clustering}

\subsection{Community detection - Optimizing the quality function}

\subsection{Louvain}
The Louvain algorithm \cite{Blondel2008b} is the state-of-the art community detection algorithm that uses an iterative greedy approach. The method can be used to optimize a partition provided by the user, but the default behaviour is to initially assign each node to its own cluster.
Each iteration is described by two repeating steps. The first step (called local moving heuristic) is to change the partition structure in a greedy manner. For each node the algorithm evaluates whether the change of its label could improve the overall quality or not. If so, the node moves to the cluster that provides the greatest increase of quality. These operations are repeated until no change is longer possible. The second step is shrinking the graph, meaning each community with a super-node. Thus, the number of nodes of the resulting graph will be the same as the number of clusters that were identified in the first step. The weight of the edges are also recalculated using the sum of inter-cluster weights of the original graph.
These two steps are repeated until the partition doesn't change after the first phase.

The algorithm can use multiple runs, when the current iterations takes as starting point the partition that is obtained in the previous one. The algorithm is said to reach convergence if no change is noticed at two consecutive iterations.

Although it is a greedy algorithm, Louvain proved it can obtain qualitative clusters. Another advantage of the method is computational efficiency: the average time complexity is $O(n \log n)$, where $n$ is the number of nodes.

% The authors claim that the order of the nodes doesn't seem to affect the final results, but can impact the time of execution.
% More details about the fact that only the labels of the neighbours are considered in the first place?

\subsection{Louvain with multilevel refinement}
Rotta and Noack \cite{Rotta2011} proposed an extension of the Louvain algorithm that introduces an extra layer of refinement. The problem the authors were trying to overcome was escaping the local optima that the first step was reaching. 

The solution they provided is to exploit a consequence of the second step of the algorithm: after the graph is shrunk and re-clustered, the clusters' composition can suffer some changes. The idea of the Louvain with multi-level refinement is to add another step after the graph is expanded back to its original structure, which is re-applying the local moving heuristic. That ensures that, at the end of each iteration, a local optima is always reached.

\subsection{SLM}
The Louvain algorithm (and its refined version) ensure local optimal node movements, but it doesn't optimize the movement of a set of nodes or the splitting of a community. Waltman and van Eck \cite{Waltman2013} proposed the Smart Local Moving algorithm (SLM). In this method, for each community that is obtained in the first step of the Louvain algorithm, the associated subgraph is generated and clustered using the local moving heuristic. If the subgraph will contain more than one cluster, the initial community will be replaced with its subgroups. This procedure enables the algorithm to move subsets of points in a local optimal way. 

One of the main differences between Louvain and SLM relates to the convergence of the results. Louvain (and its refined variant) quickly converge after a few iterations, as no community merging or node moving leads to better results. However, SLM, by splitting the communities into subclusters and moving a group of nodes, can improve its results after multiple runs. The downside is that the lack of convergence will lead to more execution time.

\subsection{Leiden}

\section{PhenoGraph pipeline}
    PhenoGraph \cite{Levine2015} is a pipeline proposed by Levine et al. to process biological data and obtain a clustering that is interpreted as different cell types. The pipeline is consists of the following steps:
    \begin{enumerate}
        \item dimensionality reduction
        \item graph construction
        \item graph clustering
    \end{enumerate}

    Each step will be described in detail in the following sections.

    \subsection{Dimensionality reduction}
    Given that the human genome contains approximately 25-30000 genes, it is expected that the input data (that is, cells extracted from a tissue from multiple donors) will be highly dimensional. Clustering techinques are highly reliant on calculating distances between points, thus an increased number of dimension will lead to expensive computations. The solution for this is to reduce the input space such that no information is lost.
    
    One of the most used approach is Principal Component Analysis (PCA) \cite{WOLD198737}, which is a method that uses linear combinations (called principal components) of the initial features to reduce the number of dimensions. This algorithm relies on computing the singular values decomposition, which is a heavy computational task, but several methods of truncating the calculation were developed in order to increase the algorithm's efficiency \cite{Baglama2016IRLBAFP}. To prevent the loss of the original information, the common practice is to use somewhere between 30 and 50 principal components.

    Dimensionality reduction can also be performed in a non-linear fashion. Here we mention UMAP \cite{mcinnes2018uniform}, an graph-based method that tries to optimize a cross-entropy function in order to create a reduced space that preserves the topology of the original data: the similar points are kept in close proximity, while maintaining the separation between distinct well-defined groups. Compared to the linear methods, UMAP manages to preserve the structure of the original data in only two or three dimensions. This characteristic makes UMAP a more suitable choice when it comes to visualising the data. The downside of the non-linear method that, given its stochastic nature, it can be affected by the value of the random seeds. Usually the effect is presented as slight changes of the topology of the groups or rotations of the representation.

    \subsection{Graph construction}
    As mentioned before, the relationship between nodes is better described in a graph structure, where the existence of the edges between nodes indicates a specific degree of similarity. The conversion is performed using the kNN alogrithm. This explains the need of performing dimensionality reduction prior to the graph construction, as the distance calculation becomes computationally expensive on large number of features.
    
    The first step is to calculate the \textit{k} nearest neighbours for each node using a distance metric (usually Euclidian or cosine). The graph will be created by drawing edges between a node and its \textit{k} neighbours. The neighbourhood relationship is not symmetrical, thus the graph will be directed. 
    
    Given the curse of dimensionality \cite{Altman2018}, distance cannot be trusted to be used for the weight of the edges. Thus, the authors of the PhenoGraph proposed to calculate the weights using the Jaccard Similarity Index (JSI) of the neighbourhoods. The weights can take values between 0 and 1 using the formula 
    
    \[ W_{ij} = \frac{|N_k(i) \cap N_k(j)|}{|N_k(i) \cup N_k(j)|}, \] where $N_k(i)$ indicates the k-neighbourhood of the node $i$. Thus, weights closer to zero indicate that the two points have significantly different neighbourhoods, whereas values closer to one mark a good overlap between the sets of neighbours.

    The weighted graph is also referred in literature as Shared Nearest Neighbour graph (SNN).
    
   % It should be mentioned that the node its included in his own neighbourhood. Otherwise, this would lead to misclassification of weights, especially in the case where the two nodes are direct neighbours, but the rest of their neighbourhoods are disjoint. If $v(i)$ doesn't include $i$, the intersection will not contain any element, thus the edge will get a weight of zero, altough the points are direct neighbours. 

   % [[Add the two photos with the examples from the supplementary]]

    \subsection{Graph clustering}
    The last step consists in identifying the clusters using a community detection algorithm. As presented in the previous section, this method is computationally efficient and manages to group points into dense subgraphs by optimizing a quality function, without being dependant on the cluster size or shape.

    The original PhenoGraph article proposed using the Louvain article, but in our experiments we will use its improvements aswell, namely Louvain with multilevel refinement, SLM and Leiden. 

    The authors' approach on getting results that are not affected by the seed was to run the community detection 100 times and choose the most qualitative partition.

    \subsection{Describing the pipeline}
    Present the steps that describe the pipeline. (Dimensionality reduction, graph building and graph clustering)
    \subsubsection{About dimensionality reduction}
    
    \subsection{How to convert matrix data into a graph using kNN}
    \subsection{SNN - providing weights using Jaccard Similarity Index}
    

\section{Element-Centric Similarity}
    \subsection{Introductory terms}
    \textit{Cluster affiliation graph}: a bipartite graph that represents the relationship between points and clusters. One vertex set will contain the points, while for the other one node is associated with a cluster. An edge indicates that a point belongs to a cluster (or, viceversa, that a cluster contains that point).
    
    \textit{Cluster-induced element graph}: a directed graph where the node set is represented by the original points. Two nodes share an edge if they belong to the same cluster.

    \textit{Affinity matrix} The affinity matrix, or the personalized PageRank (PPR) affinity is determined by calculating the paths between elements of the cluster-induced element graph. The PPR of a node $i$ to the others is computed as follows:

    \[ p_i = (1 - \alpha) v_i + \alpha p_i W \]

    where $v_i$ is a one-hot encoded vector with 1 on the $i^{th}$ position, $W$ the weighted adjacency matrix of the cluster-induced element graph and $\alpha$ is a parameter that determines the importance of overlaps in overlapping or hierarchical communities.

    For disjoint partitions, the formula is simplified to the following form:

    \[ 
        p_{ij} = 
        \begin{cases}
            0, \text{if i and j don't belong to the same cluster} \\
            \frac{\alpha}{|C_\beta|}, \text{if i and j belong to the cluster } \beta \\
            1 - \alpha + \frac{\alpha}{|C_\beta|}, \text{if } i = j
        \end{cases}
    \]
    
    \subsection{Description about how it works}

    Element-Centric similarity (ECS) is a new clustering comparison measure proposed by Gates et al \cite{Gates2019}. This score is based on calculating the L1 distance between the affinity matrices that are associated with the two clusterings:

    \[ S_i (\mathcal{A}, \mathcal{B}) = 1 - \frac{1}{2 \alpha} \sum_{j = 1}^N |p_{ij}^{\mathcal{A}} - p_{ij}^{\mathcal{B}} | \]

    A major aspect that differentiate ECS from the other metrics is that it is element-centric and not cluster-centric, that is the score is calculated individually for each element of the partition. The overall score can be calculated as the mean.

    Describe the intuition behind ECS: the idea of the bipartite graph between points and clusters.

    More details about how to calculate ECS. Talk about the affinity matrix and the L1 distance.
    \subsection{Properties, comparison with other clustering metrics}
    Present some limitation of other clustering metrics such as bias toward cluster sizes, shapes and so on. Perhaps present some comparison figures from the main article.

    Present some properties of ECS:
    \begin{enumerate}
        \item the fact that it can be used not only for flat disjoint clusterings, but also for overlapping or hierarchical partitions
        \item it overcomes the biases present in the other clustering metrics
        \item ECS illustrates the overall similarity between two partitions but also can help in identifying the points where the clustering are not similar
    \end{enumerate}
    \subsection{Element-Centric Consistency}
    Gates et. al also introduce in the same paper the concept of \textit{frustration}, which is used when a list of clusterings is provided. The frustration is calculated as the average of the pairwise similarity scores:

    \[ \frac{2}{T(T-1)} \sum_{j=1}^T \sum_{k=1}^{j-1} S_i (\mathcal{R}_k, \mathcal{R}_j), \]

    where $T$ denotes the number of clusterings. We will refer to this metric as \textit{Element-Centric Consistency} (ECC), as higher values of the score indicate a more consistent labeling of the points in the same cluster.
    

\section{Intro info about biological data and sequencing techniques}
Tell about sequencing techniques, how the initial data looks, about cells, genes, what they mean, what is the role and the purpose of the clusters in the biological interpretation.
