\documentclass[12pt]{report}

% includes
\usepackage{geometry}           % page size
\usepackage[utf8]{inputenc}     % encoding
\usepackage{palatino}           % font
\usepackage[english]{babel}    % language
\usepackage{graphicx}           % images
\usepackage{float}
\usepackage{indentfirst}        % indentation
\usepackage[nottoc]{tocbibind}  % table of contents style
\usepackage[unicode]{hyperref}  % references from the table of contents
\usepackage[font=footnotesize,labelfont=bf]{caption}
\usepackage{amsmath}
\usepackage{amssymb}
\usepackage{amsthm}
\usepackage[table,xcdraw]{xcolor}
\usepackage{tablefootnote}
\usepackage{parskip}
\usepackage[ruled,vlined,linesnumbered,resetcount]{algorithm2e}
\usepackage{pdflscape}


\graphicspath{{./images/}}

% includes options
\geometry{  a4paper,            % scientific thesis standard
            left=3cm,
            right=2cm,
            top=2cm,
            bottom=2cm,
 }
\graphicspath{{images/}}        % path where the images are located
\setlength\parindent{0pt}

% other options
\linespread{1.25}                % space between lines
\renewcommand*\contentsname{Table of contents}    % table of contents name
\newtheorem{remark}{Remark}
% the document content
\begin{document}
    % macros (global)
    \newcommand{\university}    {Universitatea "Alexandru-Ioan Cuza" din Iași}
\newcommand{\universityg}   {Universității "Alexandru-Ioan Cuza" din Iași} % genitive
\newcommand{\faculty}       {Facultatea de informatică}
\newcommand{\facultyg}      {Facultății de informatică} % genitive
\newcommand{\speciality}    {Studii Avansate în Informatică}
\newcommand{\promotion}     {2022}                                  %<---------

\newcommand{\thesistype}    {Lucrare de disertatie}
\newcommand{\thesistitle}   {Methods for assessing clustering stability in single-cell expression datasets}    %<---------

\newcommand{\authorlast}    {Munteanu}                               %<---------
\newcommand{\authorfirst}   {Andi}
\newcommand{\authornamefl}  {\authorfirst \space \authorlast} % first name first
\newcommand{\authornamelf}  {\authorlast \space \authorfirst} % last name first
\newcommand{\authorbirth}   {01 ianuarie 2018}                      %<---------
\newcommand{\authoraddress} {România, jud. Iași, mun. Iași, calea Buzăului, nr. 25, bl. A, et. 5, ap. 45} %<---------
\newcommand{\authorcnp}     {1234567891234}                         %<---------

\newcommand{\session}       {Iulie, 2022}                       %<---------
\newcommand{\coordinator}   {Conf. Dr. Liviu Ciortuz}               %<---------

\newcommand{\dottedline}    {............................}
\renewcommand{\qedsymbol}{$\blacksquare$}
\renewcommand{\footnotesize}{\scriptsize} 
    
    % front-matter
    \pagenumbering{gobble}

    \input{front/cover}
    \input{front/titlepage}
    %\input{front/declaration1}
    %\input{front/declaration2}
    % table of contents
    \tableofcontents

    % chapters
    \setcounter{page}{1}
    \pagenumbering{arabic}
    
    \chapter*{Acknowledgments} 
\addcontentsline{toc}{chapter}{Acknowledgments}

This thesis is based on the ClustAssess paper \cite{clustassess} where I contributed as co-author. All the figures used in the last three chapters are also used in the paper. The writing of the article and the development of the \verb|ClustAssess| package was done under the supervision and mentorship of professor Irina Mohorianu and Arash Shahsavari from the Core Bioinformatics Group of the Cambridge Stem Cell institute. The experiments and benchmark were also performed on a server belonging to the Core Bioinformatics group. The code of the package is made publicly available at the GitHub repository \url{https://github.com/Core-Bioinformatics/ClustAssess} and can be used under the MIT license.
    \chapter*{Introduction} 
\addcontentsline{toc}{chapter}{Introduction}

The unsupervised clustering of points remains an essential tasks in data processing and machine learning. Its relevance is underlined by the diversity of domains where clustering plays an important role. One of these relates to the bioinformatics field, where the clustering became essential in the identification process of cell types \cite{Kiselev2019a} in single-cell RNA-seq datasets or the inference of cell trajectories \cite{Saelens2019}.

As the size of the dataset grew to the size of millions of cells \cite{Svensson2020a}, these tasks have become less obvious: the cell annotations has become more reliant on the output of the clustering algorithms. The challenge was then to improve the accuracy of the classification methods as well as their performance and scalability.

One of the earliest clustering techniques is k-means \cite{Lloyd1982}, which gained popularity due to its simplicity and intuitive approach. Its bias on the size and shape of the clusters determined the introduction of other types of clustering such as density-based (DBSCAN \cite{ester1996}), hierarchical, distribution-based (EM \cite{Dempster1977}).

In the last decades the storage of data in graph-based structures has become more relevant \cite{cook2006mining}, as it was able to capture the relationship between observations. Single-cell analysis is one of the domains where graphs are the natural structure to represent the data, as it can keep the information about the relationship between cells. This lead to the introduction of a new type of clustering methods, namely the graph clustering.

While there are many approaches that were proposed to solve this problem, one of the methods that has become to be adopted at a larger scale was the community detection. This state-of-the art of this category of methods is represented by the Louvain \cite{Blondel2008b} algorithm, which takes a two-step approach to perform a greedy optimisation of an objective (also called quality in literature) function that attempts to define a good graph partitioning. The algorithm was furtherly optimised and improved in later variants such as Louvain with multi-level refinement \cite{Rotta2011}, Smart Local Moving \cite{Waltman2013} and Leiden \cite{Traag2019a}.

The limitation that single-cell analysis faced was that its default data representation structure was not a graph, but a matrix that describe the expression levels of the present cells. To address this issue, the PhenoGraph \cite{Levine2015} was introduced and proposed a pipeline of clustering the data in three steps: dimensionality reduction using approximate PCA (for example, the Lanczos bidiagonalization method \cite{Baglama2016IRLBAFP}) or UMAP \cite{mcinnes2018uniform}, graph construction using kNN \cite{Xu2015}, and graph clustering using community detection.

This pipeline was incorporated and implemented in different frameworks used specifically for processing the single-cell data, like Seurat \cite{Hao2021}, Monocle \cite{Cao2019} or SCANPY \cite{Wolf2018}. In this thesis we will compare how the PhenoGraph pipeline was included in the Seurat and Monocle packages and will analyze the main technical sources that lead to divergent results between them.

This analysis raised the question of the pipeline stability when the random seed changes, as most of the algorithms involved contain at least one stochastic component. The instability cause by random seed was issued in previous works. Some of the approaches were to modify the algorithm as in kmeans++ \cite{kmeanspp} or to add noise to data as in the clust-perturb algorithm \cite{STACEY2021}. We propose \verb|ClustAssess|, a R package that provides a pipeline meant to visually guide the user into choosing a configuration of parameters that leads to results where the seed effect is negligible, without performing any changes to the methods or the original data.

The stability is inferred by running the pipeline multiple times with different seeds and compare the results using the Element-Centric Similarity score \cite{Gates2019}. This score is a clustering comparing tool that does not have any bias regarding the size of the clusters, their shape or the problem of matching, like the traditional measurements (such as NMI \cite{McDaid2015} or ARI \cite{Collins1988}). Our package also provides an optimized implementation of the ECS score that scales well on larger datasets.


    
    \chapter{Description of methods}

This chapter contains informations about the methods used for graph clustering, the sequencing and processing the biological data and eventualy mentions of other works / papers that were focusing on assessing the robustness on changing the seed.

\section{Graph Clustering}

\subsection{Short intro about what graph clustering is}

\subsection{Why graph clustering instead other traditional methods such as k-means, density based techniques etc}

\subsection{Types of graph clustering}

\subsection{Community detection - Optimizing the quality function}

\subsection{Louvain}
The Louvain algorithm \cite{Blondel2008b} is the state-of-the art community detection algorithm that uses an iterative greedy approach. The method can be used to optimize a partition provided by the user, but the default behaviour is to initially assign each node to its own cluster.
Each iteration is described by two repeating steps. The first step is to change the partition structure in a greedy manner. For each node the algorithm evaluates whether the change of its label could improve the overall quality or not. If so, the node moves to the cluster that provides the greatest increase of quality. This operations is repeated until no change is longer possible. The second step is shrinking the graph, meaning each community with a super-node. Thus, the number of nodes of the resulting graph will be the same as the number of clusters that were identified in the first step. The weight of the edges are also recalculated using the sum of inter-cluster weights of the original graph.
These two steps are repeated until the partition doesn't change after the first phase.

The algorithm can use multiple runs, when the current iterations takes as starting point the partition that is obtained in the previous one. The algorithm is said to reach convergence if no change is noticed at two consecutive iterations.

Altough it is a greedy algorithm, Louvain proved it can obtain qualitative clusters. Another advantage of the method is computational efficiency: the average time complexity is $O(n \log n)$, where $n$ is the number of nodes.

% The authors claim that the order of the nodes doesn't seem to affect the final results, but can impact the time of execution.
% More details about the fact that only the labels of the neighbours are considered in the first place?

\subsection{Louvain refined}
\subsection{SLM}
\subsection{Leiden}

\section{PhenoGraph pipeline}
    PhenoGraph \cite{Levine2015} is a pipeline proposed by Levine et al. to process biological data and obtain a clustering that is interpreted as different cell types. The pipeline is consists of the following steps:
    \begin{enumerate}
        \item dimensionality reduction
        \item graph construction
        \item graph clustering
    \end{enumerate}

    Each step will be described in detail in the following sections.

    \subsection{Dimensionality reduction}
    Given that the human genome contains approximately 25-30000 genes, it is expected that the input data (that is, cells extracted from a tissue from multiple donors) will be highly dimensional. Clustering techinques are highly reliant on calculating distances between points, thus an increased number of dimension will lead to expensive computations. The solution for this is to reduce the input space such that no information is lost.
    
    One of the most used approach is Principal Component Analysis (PCA) \cite{WOLD198737}, which is a method that uses linear combinations (called principal components) of the initial features to reduce the number of dimensions. This algorithm relies on computing the singular values decomposition, which is a heavy computational task, but several methods of truncating the calculation were developed in order to increase the algorithm's efficiency \cite{Baglama2016IRLBAFP}. To prevent the loss of the original information, the common practice is to use somewhere between 30 and 50 principal components.

    Dimensionality reduction can also be performed in a non-linear fashion. Here we mention UMAP \cite{mcinnes2018uniform}, an graph-based method that tries to optimize a cross-entropy function in order to create a reduced space that preserves the topology of the original data: the similar points are kept in close proximity, while maintaining the separation between distinct well-defined groups. Compared to the linear methods, UMAP manages to preserve the structure of the original data in only two or three dimensions. This characteristic makes UMAP a more suitable choice when it comes to visualising the data. The downside of the non-linear method that, given its stochastic nature, it can be affected by the value of the random seeds. Usually the effect is presented as slight changes of the topology of the groups or rotations of the representation.

    \subsection{Graph construction}
    As mentioned before, the relationship between nodes is better described in a graph structure, where the existence of the edges between nodes indicates a specific degree of similarity. The conversion is performed using the kNN alogrithm. This explains the need of performing dimensionality reduction prior to the graph construction, as the distance calculation becomes computationally expensive on large number of features.
    
    The first step is to calculate the \textit{k} nearest neighbours for each node using a distance metric (usually Euclidian or cosine). The graph will be created by drawing edges between a node and its \textit{k} neighbours. The neighbourhood relationship is not symmetrical, thus the graph will be directed. 
    
    Given the curse of dimensionality \cite{Altman2018}, distance cannot be trusted to be used for the weight of the edges. Thus, the authors of the PhenoGraph proposed to calculate the weights using the Jaccard Similarity Index (JSI) of the neighbourhoods. The weights can take values between 0 and 1 using the formula 
    
    \[ W_{ij} = \frac{|v(i) \cap v(j)|}{|v(i) \cup v(j)|}, \] where $v(i)$ indicates the neighbourhood of the node $i$. Thus, weights closer to zero indicate that the two points have significantly different neighbourhoods, whereas values closer to one mark a good overlap between the sets of neighbours.
    
    It should be mentioned that the node its included in his own neighbourhood. Otherwise, this would lead to misclassification of weights, especially in the case where the two nodes are direct neighbours, but the rest of their neighbourhoods are disjoint. If $v(i)$ doesn't include $i$, the intersection will not contain any element, thus the edge will get a weight of zero, altough the points are direct neighbours. 

    [[Add the two photos with the examples from the supplementary]]


    Comp

    \subsection{Graph clustering}
    The last step consists in identifying the clusters using a community detection algorithm. As presented in the previous section, this method is computationally efficient and manages to group points into dense subgraphs by optimizing a quality function, without being dependant on the cluster size or shape.

    The original PhenoGraph article proposed using the Louvain article, but in our experiments we will use its improvements aswell, namely Louvain with multilevel refinement, SLM and Leiden. 

    The authors' approach on getting results that are not affected by the seed was to run the community detection 100 times and choose the most qualitative partition.

    \subsection{Describing the pipeline}
    Present the steps that describe the pipeline. (Dimensionality reduction, graph building and graph clustering)
    \subsubsection{About dimensionality reduction}
    
    \subsection{How to convert matrix data into a graph using kNN}
    \subsection{SNN - providing weights using Jaccard Similarity Index}
    

\section{Element-Centric Similarity}
    \subsection{Introductory terms}
    \textit{Cluster affiliation graph}: a bipartite graph that represents the relationship between points and clusters. One vertex set will contain the points, while for the other one node is associated with a cluster. An edge indicates that a point belongs to a cluster (or, viceversa, that a cluster contains that point).
    
    \textit{Cluster-induced element graph}: a directed graph where the node set is represented by the original points. Two nodes share an edge if they belong to the same cluster.

    \textit{Affinity matrix} The affinity matrix, or the personalized PageRank (PPR) affinity is determined by calculating the paths between elements of the cluster-induced element graph. The PPR of a node $i$ to the others is computed as follows:

    \[ p_i = (1 - \alpha) v_i + \alpha p_i W \]

    where $v_i$ is a one-hot encoded vector with 1 on the $i^{th}$ position, $W$ the weighted adjacency matrix of the cluster-induced element graph and $\alpha$ is a parameter that determines the importance of overlaps in overlapping or hierarchical communities.

    For disjoint partitions, the formula is simplified to the following form:

    \[ 
        p_{ij} = 
        \begin{cases}
            0, \text{if i and j don't belong to the same cluster} \\
            \frac{\alpha}{|C_\beta|}, \text{if i and j belong to the cluster } \beta \\
            1 - \alpha + \frac{\alpha}{|C_\beta|}, \text{if } i = j
        \end{cases}
    \]
    
    \subsection{Description about how it works}

    Element-Centric similarity is a new clustering comparison measure proposed by Gates et al \cite{Gates2019}. This score is based on calculating the L1 distance between the affinity matrices that are associated with the two clusterings:

    \[ S_i (\mathcal{A}, \mathcal{B}) = 1 - \frac{1}{2 \alpha} \sum_{j = 1}^N |p_{ij}^{\mathcal{A}} - p_{ij}^{\mathcal{B}} | \]

    Describe the intuition behind ECS: the idea of the bipartite graph between points and clusters.

    More details about how to calculate ECS. Talk about the affinity matrix and the L1 distance.
    \subsection{Properties, comparison with other clustering metrics}
    Present some limitation of other clustering metrics such as bias toward cluster sizes, shapes and so on. Perhaps present some comparison figures from the main article.

    Present some properties of ECS:
    \begin{enumerate}
        \item the fact that it can be used not only for flat disjoint clusterings, but also for overlapping or hierarchical partitions
        \item it overcomes the biases present in the other clustering metrics
        \item ECS illustrates the overall similarity between two partitions but also can help in identifying the points where the clustering are not similar
    \end{enumerate}
    \subsection{ECC}
    Talk about how ECC is calculated

\section{Intro info about biological data and sequencing techniques}
Tell about sequencing techniques, how the initial data looks, about cells, genes, what they mean, what is the role and the purpose of the clusters in the biological interpretation.


    \chapter{The importance of parameter values in the clustering output}

\section{Monocle and Seurat}
As stated in the previous chapter, the PhenoGraph pipeline was widely adopted in the task of processing biological data, due to its efficiency and quality of results. The pipeline was incorporated and implemented in various programming languages.

The most frequently used R packages that perform single-cell data processing are Seurat (currently on the third version) \cite{Hao2021} and Monocle (currently at the third version) \cite{Cao2019}. Both packages use the PhenoGraph method for the cell clustering. The goal of this chapter is to evaluate the output of the two packages and compare the results side by side.

\section{Data used for experiments}
To perform this comparison, we used the Cuomo data \cite{Cuomo2020}, an in-vitro SMART-seq dataset of human endoderm differentiation. The dataset contains 1880 cells extracted from six donors (hayt, naah, vils, pahc, melw and qunz) at four different timepoints. The cells are in expressed in approximately 30 000 genes. The dataset was preprocessed by removing the cells that are expressed in less than 200 features and the genes that are expressed in less than three cells. MT (mitochondrial) and RP (ribosomal protein) genes were also excluded from the feature set.

The purpose of this thesis is to analyze the clustering pipeline, therefore the preprocessing parameters (normalization and scaling) are done in the same manner for both packages.

\section{Algorithms used in the pipeline}
In chapter one we presented the steps that are involved in the PhenoGraph pipeline, namely the dimensionality reduction, the graph construction and the community detection. Each step can be performed by a varying number of algorithms. Therefore, in order to compare the two R packages, we must identify the algorithms that they are using in the clustering pipeline (see Table \ref{tab:s4-m3-methods}).

For the dimensionality reduction, both packages allow the usage of either linear methods (PCA) or non-linear ones (UMAP and tSNE). The developers of the both packages recommend performing the PCA reduction on the raw data, followed by tSNE or UMAP. The graph construction is performed using the kNN based method. For the community detection, Monocle uses the Louvain and Leiden alogrithms, whereas Seurat can cluster the cells using Louvain with multi-level refinement and SLM, too.

We note that all the algorithms involved in the clustering pipeline contain at least one stochastic component, therefore 

\begin{table}[]
    \begin{tabular}{|l|l|l|l|}
        \hline
                         & \textbf{Dim reduction}    & \textbf{Graph construction} & \textbf{Graph clustering} \\ \hline
        \textbf{Seurat}  & \begin{tabular}[c]{@{}l@{}}PCA followed\\ by tSNE or UMAP\end{tabular} & \begin{tabular}[c]{@{}l@{}}kNN based\\ support for SNN\end{tabular}   & \begin{tabular}[c]{@{}l@{}}Louvain, Louvain refined,\\ SLM, Leiden\end{tabular} \\ \hline
        \textbf{Monocle} & \begin{tabular}[c]{@{}l@{}}PCA followed\\ by tSNE or UMAP\end{tabular} & \begin{tabular}[c]{@{}l@{}}kNN based\\ support for SNN\end{tabular}   & Louvain, Leiden           \\ \hline
    \end{tabular}
    \caption{\label{tab:s4-m3-methods}The algorithms used by Monocle and Seurat inside the Phenograph pipeline}
\end{table}

\section{Comparing the results}

\begin{figure}[H]
    \centering
    \includegraphics[width=7cm]{2_S1.png}
    \includegraphics[width=7cm]{2_M1.png}
    \caption{\label{fig:s4-m3-default}Clustering distribution with default parameters for Monocle (left) and Seurat (right). The title indicates the default random seed that is used.}
\end{figure}

If the Cuomo \cite{Cuomo2020} dataset is clustered using the Monocle and Seurat packages with default parameters, significantly different outputs are produced, as it can be observed in Figure \ref{fig:s4-m3-default}. Firstly, there are differences with respect to the dimensionality reduction. The cells are scattered in multiple islands in the Monocle package, whereas Seurat obtains a more compact grouping. Also, the two packages disagree regarding the number of clusters: Seurat outputs 13 clusters, Monocle 15.

The biological interpretation of the data heavily relies on the clusters' structure. Thus, is expected for the discrepancies between the two packages to extend over the marker identification and the biological interpretation. The question we pose is whether the divergence between Monocle and Seurat has technical (such as the implementation of the clustering pipeline) or biological (such as additional pre- or post-processing steps that rely on the sequencing information).

\section{Aligning the results}





    \chapter{Assessing the stability}

The results of the previous chapter highlighted not only the impact of the parameters on the clustering output, but also raised the question of interpretation stability. While there are parameters which are designed to affect the clustering output, such as the resolution parameter, the community detection, the objective function or the feature space, there are some factors that should not be responsible for obtaining different results.

One if these factors is the random seed. It is not an expected behaviour to get different partitions and different downstream biological interpretations if only the seed is changed.

\section{ClustAssess}
Under the supervision of professor Irina Mohorianu and Arash Shahsavari, I was part of the team that developed the R package \verb|ClustAssess| \footnote{GitHub official repository: \url{https://github.com/Core-Bioinformatics/ClustAssess}} \cite{clustassess}.

The package has several functionalities, including:
\begin{itemize}
    \item determine the optimal number of clusters using PAC (method based on cumulative distribution)
    \item calculate the Element-Centric Similarity (ECS) between two flat, overlapping or hierarchical partitions
    \item calculate the Element-Centric Consistency (ECC) of a list containing flat, overlapping and hierarchical partitions
    \item calculate the marker gene similarity across each cell
    \item stability-based parameter assessment pipeline.
\end{itemize}
This chapter will contain a description of the package components that I contributed on research and developing, namely calculating the ECS score and creating the stability pipeline.

\section{ECS calculation}
As far as we are concerned, the Python package \verb|CluSim| \footnote{GitHub official repository: \url{https://github.com/Hoosier-Clusters/clusim}} \cite{Gates2019b}, which is developed and maintained by the authors of the ECS paper \cite{Gates2019}, is the only official source that contains an implementation of the Element-Centric Similarity measure.


The \verb|ClustAssess| package offers an R implementation of this score. The implementation follows the steps presented in the article presented above. For each partition, the cluster affiliation graph is determined, followed by calculating the cluster-induced element graph and the affinity matrix. Once the affinity matrices are calculated, the ECS can be calculated as the L1 distance between them.

\subsection{Optimising the calculation of the ECS}
However, as stated in the first chapter, computing the PPR matrix for flat disjoint partitions can be done using the formula \ref{eq:affinity-disjoin}. As it can be observed, for each pair of points, there can only be three values, all of them being independent on the nature of the points. We will use this observation to prove that the calculation of the ECS score can be optimised.



Let $\mathcal{A}$ and $\mathcal{B}$ be two partitions of the same set of points $\mathcal{P}$. Let $C_a^\mathcal{A}$ denote the $\text{a}^{th}$ cluster of the partition $\mathcal{A}$, and $C_b^\mathcal{B}$ denote the $\text{b}^{th}$ cluster of the partition $\mathcal{B}$. Denote the length of the clusters as follows: $|C_a^{\mathcal{A}}| = c_a \text{ and } |C_b^{\mathcal{B}}| = c_b$.

\begin{remark} \label{remark:pii}
    $p_{ii}^\mathcal{A} - p_{ii}^\mathcal{B} = p_{ij}^\mathcal{A} - p_{ij}^\mathcal{B}$, for any $i, j \in C_a^\mathcal{A} \cap C_b^\mathcal{B}$
\end{remark}

\begin{proof}
    The proof is immediate when $i = j$.

    For $i \neq j$, from the Equation \ref{eq:affinity-disjoin}, we will have:
    $ 
        \displaystyle
        \begin{cases}
            p_{ij}^\mathcal{A} &= \frac{\alpha}{c_a} \\
            p_{ij}^\mathcal{B} &= \frac{\alpha}{c_b} 
        \end{cases}
    $.
    Therefore, 
    \begin{equation} \label{eq:rem1-i}
        p_{ij}^\mathcal{A} - p_{ij}^\mathcal{B} = \alpha\left(\frac{1}{c_a} - \frac{1}{c_b}\right) 
    \end{equation}

    From the same equation, we get that 
    $ 
        \displaystyle
        \begin{cases}
            p_{ii}^\mathcal{A} &= 1-\alpha + \frac{\alpha}{c_a} \\
            p_{ii}^\mathcal{B} &= 1-\alpha + \frac{\alpha}{c_b} 
        \end{cases}
    $. Therefore, 
    \begin{equation} \label{eq:rem1-ii}
        p_{ii}^\mathcal{A} - p_{ii}^\mathcal{B} = \left(1-\alpha + \frac{\alpha}{c_a}\right) - \left(1-\alpha+\frac{\alpha}{c_b}\right) = \alpha\left(\frac{1}{c_a}-\frac{1}{c_b}\right)
    \end{equation}

    From \ref{eq:rem1-i} and \ref{eq:rem1-ii}, we get that $p_{ii}^\mathcal{A} - p_{ii}^\mathcal{B} = p_{ij}^\mathcal{A} - p_{ij}^\mathcal{B}$.
\end{proof}

\begin{remark} \label{remark:ecs-constant}
    $\displaystyle\mathcal{S}_i(\mathcal{A}, \mathcal{B}) = 1-\frac{1}{2}\left(|C_a^\mathcal{A} \cap C_b^\mathcal{B}|\cdot \left|\frac{1}{c_a} - \frac{1}{c_b}\right| + |C_a^\mathcal{A} \cap (\mathcal{P} \setminus C_b^{\mathcal{B}})| \cdot \frac{1}{c_a} + |(\mathcal{P} \setminus C_a^\mathcal{A}) \cap C_b^\mathcal{B}|\cdot \frac{1}{c_b} \right)$, for any $i \in C_a^\mathcal{A} \cap C_b^\mathcal{B}$.
\end{remark}

\begin{proof}
   In order to calculate the Element-Centric Similarity between partitions $\mathcal{A}$ and $\mathcal{B}$ in the point $i$, we must use the formula as described in the Equation \ref{eq:def-ecs}.

   To define $p_{ij}^{\mathcal{A}}$ and $p_{ij}^{\mathcal{B}}$, we must explore all four cases of labeling the point $j$. 
    
   \textbf{Case 1:} $j \in C_a^{\mathcal{A}} \cap C_b^{\mathcal{B}}$ ($j$ belongs to the same cluster as $i$ in both partitions.)

   Using the formula from \ref{eq:affinity-disjoin}, we have $\displaystyle
    \begin{cases}
        p_{ij}^\mathcal{A} &= \frac{\alpha}{c_a} \\
        p_{ij}^\mathcal{B} &= \frac{\alpha}{c_b} 
    \end{cases}$. Therefore, 
    \[ p_{ij}^\mathcal{A} - p_{ij}^\mathcal{B} = \alpha \left(\frac{1}{c_a} - \frac{1}{c_b}\right) .\]

    This equation also holds when $j = i$, as proven in Remark \ref{remark:pii}.

    \textbf{Case 2:} $j \in C_a^\mathcal{A} \cap (\mathcal{P} \setminus C_b^{\mathcal{B}})$ ($j$ belongs to the same cluster as $i$ only in the first partition.)

   Using the formula from \ref{eq:affinity-disjoin}, we have $\displaystyle
    \begin{cases}
        p_{ij}^\mathcal{A} &= \frac{\alpha}{c_a} \\
        p_{ij}^\mathcal{B} &= 0
    \end{cases}$. Therefore, 
    \[ p_{ij}^\mathcal{A} - p_{ij}^\mathcal{B} = \alpha \frac{1}{c_a} .\]

    \textbf{Case 3:} $j \in (\mathcal{P} \setminus C_a^\mathcal{A}) \cap  C_b^{\mathcal{B}}$ ($j$ belongs to the same cluster as $i$ only in the second partition.)

   Using the formula from \ref{eq:affinity-disjoin}, we have $\displaystyle
    \begin{cases}
        p_{ij}^\mathcal{A} &= 0 \\
        p_{ij}^\mathcal{B} &=\frac{\alpha}{c_b}
    \end{cases}$. Therefore, 
    \[ p_{ij}^\mathcal{A} - p_{ij}^\mathcal{B} = \alpha \frac{1}{c_b} .\]

    \textbf{Case 4:} $j \in (\mathcal{P} \setminus C_a^\mathcal{A}) \cap  (\mathcal{P} \setminus C_b^{\mathcal{B}})$ ($j$ belongs to a different cluster than $i$ both partitions.)

   Using the formula from \ref{eq:affinity-disjoin}, we have $\displaystyle
    \begin{cases}
        p_{ij}^\mathcal{A} &= 0 \\
        p_{ij}^\mathcal{B} &=0
    \end{cases}$. Therefore, 
    \[ p_{ij}^\mathcal{A} - p_{ij}^\mathcal{B} = 0 .\]


    These four cases are exhaustive so $\displaystyle \sum_{j = 1}^N |p_{ij}^{\mathcal{A}} - p_{ij}^{\mathcal{B}} |$ will be equivalent to % according to \ref{eq:def-ecs}, $\mathcal{S}_i(\mathcal{A}, \mathcal{B})$ will be:

    \[
        \begin{aligned}
            &\sum_{j \in C_a^{\mathcal{A}} \cap C_b^{\mathcal{B}}} |p_{ij}^{\mathcal{A}} - p_{ij}^{\mathcal{B}} | + \sum_{j \in C_a^\mathcal{A} \cap (\mathcal{P} \setminus C_b^{\mathcal{B}})} |p_{ij}^{\mathcal{A}} - p_{ij}^{\mathcal{B}} | + \sum_{j \in (\mathcal{P} \setminus C_a^\mathcal{A}) \cap  C_b^{\mathcal{B}}} |p_{ij}^{\mathcal{A}} - p_{ij}^{\mathcal{B}} | + \sum_{j \in (\mathcal{P} \setminus C_a^\mathcal{A}) \cap  (\mathcal{P} \setminus C_b^{\mathcal{B}})} |p_{ij}^{\mathcal{A}} - p_{ij}^{\mathcal{B}} | = \\
            %
            = &\sum_{j \in C_a^{\mathcal{A}} \cap C_b^{\mathcal{B}}} \left|\alpha\left(\frac{1}{c_a} - \frac{1}{c_b}\right)\right| + \sum_{j \in C_a^\mathcal{A} \cap (\mathcal{P} \setminus C_b^{\mathcal{B}})} \left|\alpha\frac{1}{c_a}\right| + \sum_{j \in (\mathcal{P} \setminus C_a^\mathcal{A}) \cap  C_b^{\mathcal{B}}} \left|\alpha\frac{1}{c_b}\right| + \sum_{j \in (\mathcal{P} \setminus C_a^\mathcal{A}) \cap  (\mathcal{P} \setminus C_b^{\mathcal{B}}} |0 | = \\
            %
            = &\alpha\sum_{j \in C_a^{\mathcal{A}} \cap C_b^{\mathcal{B}}} \left|\frac{1}{c_a} - \frac{1}{c_b}\right| + \alpha\sum_{j \in C_a^\mathcal{A} \cap (\mathcal{P} \setminus C_b^{\mathcal{B}})} \left|\frac{1}{c_a}\right| + \alpha\sum_{j \in (\mathcal{P} \setminus C_a^\mathcal{A}) \cap  C_b^{\mathcal{B}}} \left|\frac{1}{c_b}\right| = \\
            %
            = &\alpha\left( |C_a^\mathcal{A} \cap C_b^\mathcal{B}|\cdot \left|\frac{1}{c_a} - \frac{1}{c_b}\right| + |C_a^\mathcal{A} \cap (\mathcal{P} \setminus C_b^{\mathcal{B}})| \cdot \frac{1}{c_a} + |(\mathcal{P} \setminus C_a^\mathcal{A}) \cap C_b^\mathcal{B}|\cdot \frac{1}{c_b} \right)
        \end{aligned}     
    \]

    Replacing this result in the formula \ref{eq:def-ecs} will lead to
    \[
    \begin{aligned}
        S_i (\mathcal{A}, \mathcal{B}) &= 1 - \frac{1}{2 \alpha} \sum_{j = 1}^N |p_{ij}^{\mathcal{A}} - p_{ij}^{\mathcal{B}} |  \\
        &= 1 - \frac{1}{2\alpha}\alpha\left( |C_a^\mathcal{A} \cap C_b^\mathcal{B}|\cdot \left|\frac{1}{c_a} - \frac{1}{c_b}\right| + |C_a^\mathcal{A} \cap (\mathcal{P} \setminus C_b^{\mathcal{B}})| \cdot \frac{1}{c_a} + |(\mathcal{P} \setminus C_a^\mathcal{A}) \cap C_b^\mathcal{B}|\cdot \frac{1}{c_b} \right) \\
        &= 1 - \frac{1}{2}\left( |C_a^\mathcal{A} \cap C_b^\mathcal{B}|\cdot \left|\frac{1}{c_a} - \frac{1}{c_b}\right| + |C_a^\mathcal{A} \cap (\mathcal{P} \setminus C_b^{\mathcal{B}})| \cdot \frac{1}{c_a} + |(\mathcal{P} \setminus C_a^\mathcal{A}) \cap C_b^\mathcal{B}|\cdot \frac{1}{c_b} \right)
    \end{aligned} ,
    \]
    which is the desired output.
\end{proof}

The statement of the Remark \ref{remark:ecs-constant} leads to the immediate conclusion that the ECS score is the same for points that have the same cluster labels for both partitions. This observation is important, as it indicates that the number of unique ECS values is $|\mathcal{A}| \cdot |\mathcal{B}|$, where $|\mathcal{A}|$ and $|\mathcal{B}|$ refer to the number of clusters of the partitions.

Given that the number of points grows exponentially faster than the number of clusters, it is more efficient to calculate the unique ECS values rather than to perform the same computations for all pairs of points.

In \verb|ClustAssess| we implemented this optimized version of calculating the Element-Centric similarity, which brings a significant speed-up, as well as a low-memory usage, as it will be seen in Chapter 4.

\subsection{Weighted ECC}
There is no guarantee that the list of partitions used for calculating the Element-Centric Consistency will not contain duplicates. This is can be met in different scenarios, such as calculating the consistency of the results that multiple clustering algorithms produce.

Using the original approach will lead to redundant calculations: having a list $\mathcal{L}$ of $n$ partitions, $x$ of them being duplicates, will lead to $(x-1) \cdot (n-1)$ computation of the ECS score that have already been done. Also, this allows calculating the ECS between a partition and its duplicate, whose result is already known to be 1.

This can be solved by attaching a weight to each partition to indicate the number of duplicates. This way, the ECS is calculated once and multiplied times the number of combination between the number of duplicates: $\mathcal{S}(\mathcal{L}_i, \mathcal{L}_j) \cdot w_i \cdot w_j$, where $w$ denotes the weights array. The procedure is detailed in Algorithm \ref{alg:weighted-ecc}. 

\begin{algorithm}[h!] 
    \SetKwInOut{Input}{input}\SetKwInOut{Output}{output}
    \SetKwInOut{Parameters}{parameters}
    \Input{$partList$: the list of disjoint partitions; each partition is represented as an array of labels; the partitions should have the same length. \\ $weights$: the weight array that contains the number of duplicates \\ associated to each partition.}
    \Output{the Element-Centric Consistency of the list}

    $nPartitions \gets \text{size}(partList)$; \\
    $nTotalPartitions \gets \text{sum}(weights)$; \tcp*{includes duplicates}
    $ecc \gets \textbf{0}$; \\
    \tcp{Calculate the ECS between different partitions}
    \For{i = 1 \KwTo nPartitions - 1}{
        \For{j = i + 1 \KwTo nPartitions}{
            $ecc \gets ecc + \text{ecs}(partList[i],partList[j]) * weights[i] * weights[j];$
        }
    }
    
    \tcp{Calculate the ECS between duplicates}
    $nDuplicatesECS \gets 0$; \\
    \For{i = 1 \KwTo nPartitions}{$nDuplicatesECS \gets nDuplicatesECS + weights[i] * (weights[i] - 1) / 2;$}
    $ecc \gets ecc + nDuplicatesECS * \textbf{1};$ \\
    \tcp{Normalize the score in the range [0, 1]}
    $ecc \gets ecc / (nTotalPartitions * (nTotalPartitions - 1) / 2);$ \\
    \Return{ecc}
    \caption{Weighted Element-Centric Consistency}
    \label{alg:weighted-ecc}
\end{algorithm}

\subsection{Partition merging} \label{sec:part-merge}
Although the weighted version of the ECC score is more efficient, it might be difficult for the user to detect the duplicates and create the weights array.

In the \verb|ClustAssess| package we developed the ECC calculation method such that it automatically identifies the partitions that are identical and merges them. The merge operation involves removing the duplicates and updating the weights array.

To identify the identical partitions, we chose to use a method that should be more efficient than calculating the ECS and checking whether the score is one or not. Given two partitions, we calculate their contingency table. A contingency table is a matrix where each row indicates a cluster from the first partition and each column a cluster from the second one. The value at the index $[i, j]$ indicates the number of elements that are in the $i^\text{th}$ cluster from the first partition and in the $j^\text{th}$ group from the second one simultaneously.

To determine if the given clusterings match, we must verify if there is any row or column which has more than one entry with non-zero values. If this is the case, that translates to an imperfect match between the two groups and subsequently to the conclusion that the partitions are different.

\subsubsection{ECS threshold}
There are numerous cases when two partitions do not perfectly match, but are virtually identical: from thousands of points, only a few of them are labelled in different clusters. We can visually understand this by looking at the panels S17 and M18 from Figure \ref{fig:s4-m3-res}. They look like they are the same partition, but the contingency table from Figure \ref{fig:s4-m3-cont} indicates that there are three places of imperfect matching.

This poses the question whether these partitions should be considered different or identical when calculating the ECC. To answer this issue, we introduced an additional parameter named \textit{ECS threshold}, which allows relaxing the condition of identifying a partition as duplicate of another.

The difference from the identical matching is that, instead of looking for perfect contingency tables, we calculate the mean ECS between the partitions. If the value is above the threshold, we perform the same steps as the ones from the previous case.

\subsection{Parallelization support} \label{sec:paral}
It can be noticed that the calculation of the ECS between every possible pair from the partition list can be done independently, which opens up the possibility to perform this operation concurrently.

To allow parallel execution of the ECS calculation we used the R \verb|foreach| packge \footnote{GitHub official repository: \url{https://github.com/RevolutionAnalytics/foreach}}. Using the PSOCK option, more R instances that are independent from the original process and that can run simoultaneously. Each R instance recieves the partition list and the pairs for which the ECS must be calculated, and afterwards the processes are launched in parallel. The results are summed and the mean is calculated as in the sequential case.

\section{Stability assessment pipeline}
Another major contribution brought by this package is the creation of a pipeline that evaluates the stability of the clustering.

We note that the purpose of the pipeline is not to identify optimal values of different parameters, but to provide visual tools that the user can use in order to assess the stability of different parameters involved in the clustering.

As we mentioned at the beginning of the chapter, our expectation is to achieve similar results when only the random seed is changed. Thus, we define cluster stability as the robustness of the clustering output at the change of seed values. The stability will be measured by running the clustering pipeline multiple times with different random seeds. The Element-Centric Consistency score is then calculated and used as an indicator for robustness: the confidence in the robustness at seed changes grows proportionate with the consistency of the list of the obtained partitions.

The stability assessment pipeline follows the PhenoGraph \cite{Levine2015} algorithm. The code is wrapped on the Seurat implementation of the algorithm (see Chapter 2). Although this pipeline is targeting the assessment of biological data, the PhenoGraph algorithm is agnostic to the origin of the data, therefore the pipeline could be adjusted to more general use-cases.

The stability pipeline performs the assessment of the stability of some of the parameters presented in the previous chapter that have an impact over the clustering output. The following sections describe how the assessment is performed and follow the three-step structure of the PhenoGraph algorithm. We will use the Cuomo \cite{Cuomo2020} dataset as an example of using the pipeline. A more code-oriented presentation is made available in the \href{https://core-bioinformatics.github.io/ClustAssess/articles/stability-based-parameter-assessment.html}{following vignette} that is part of the \verb|ClustAssess| package.


\subsection{Assessing the dimensionality reduction}
On this step we solely focus on the feature set parameter of PCA. Although it is not the purpose of the function, it could be used to assess the stability of other parameters such as the number of principle components. The user can provide any expression matrix that corresponds to a dataset, as well as a list with different feature types (such as most abundant genes, highly variable genes) of varying sizes. The package creates four types of plots:

\subsubsection{Stability distribution}
The plot illustrates the boxplot distribution of the ECC score of the partitions obtained when changing the seed. The boxplots are grouped based on the number of feature types: each set will be associated with a different colour. The number of groups is decided by the number of the subset sizes. Figure \ref{fig:ca-feat-inc} illustrates an example of Stability distribution of the results obtained on the Cuomo dataset after 30 runs. There are three feature types: most abundant genes (MA), highly variable genes (HV) and the intersection between them (MA\_HV). All of them have six different subset sizes: 500, 1000, 1500, 2000, 25000 and 3000 for MA and HV, and 14, 39, 77, 112, 157 and 197 for MA\_HV. 

The purpose of the plot is to identify the most consistent set of genes. In our case, MA\_HV is the most unstable set, and MA or HV with 3000 genes can be considered as suitable choices for getting robust results.
\begin{figure}[H]
    \centering
    \makebox[\textwidth][c]{\includegraphics[width=0.8\linewidth]{images/ch3/3_feat_incremental.png}}
    \caption{\label{fig:ca-feat-inc} \textbf{Assessment of Feature stability, summarised as boxplots, obtained on 30 runs.} The colours indicates different feature sets, used for calculating the PCA: MA - most abundant genes (red); HV - highly variable genes (green); MA\_HV - intersection of MA and HV genes (blue). Above each boxplot we indicate the number of features within the subset. For this dataset, the MA and HV gene sets on 3000 genes are preferred due to their high consistency across the runs.}
\end{figure}

\subsubsection{Incremental similarity}
An important task in the data processing is identifying the noise. In single-cell analysis, this task translates into identifying the noisy genes, namely the features not only do not contribute to cluster identification, but also can perturb the output. This plot relates to this tasks and shows the distribution of the ECS between the partitions that are obtained using the same feature set, but with different sizes. The grouping is done in a similar fashion with the one from the previous plot. Our approach is to translate incremental dissimilarity as an indicator of not reaching the noise. The goal is to identify the transition from the signal to the noise zone.

Figure \ref{fig:ca-feat-comp} shows the Incremental similarity plot for the Cuomo data. As the title suggests, the comparison is done incrementally: first we compare the results obtained with 500 features with the ones when using 1000, then compare 1000 with 1500 and so on. It can be noticed that for MA\_HV the noise zone has not been reached, as the last blue boxplot suggests that adding the last genes has a significant impact upon the final clustering. As for the other two feature types, the high distribution of similarity in the last two groups of boxplots suggests that using more than 3000 genes would not lead to any information gain.

\begin{figure}[H]
    \centering
    \makebox[\textwidth][c]{\includegraphics[width=0.8\linewidth]{images/ch3/3_feat_comparative.png}}
    \caption{\label{fig:ca-feat-comp}\textbf{Assessment of reaching the noise level using incremental similarity, obtained on 30 runs.} The colours indicates different feature sets, used for calculating the PCA: MA - most abundant genes (red); HV - highly variable genes (green); MA\_HV - intersection of MA and HV genes (blue). Above each boxplot we indicate the number of features within the subset. Each boxplot indicates the distribution of the ECS between the most frequent partitions obtained with incremental sizes. For this dataset, we notice no further changes for MA and HV sets at 2500-3000 genes, meaning the noise level was reached.}
\end{figure}

\subsubsection{Distribution of clusters}
The last two plots provide information about the number of genes that should be used and the consistency of the feature set, but the consistency can have multiple causes. One factor that can influence the stability is the topology of the data. The following plot illustrates the cluster distribution for each subset of genes on the UMAP space.

An example of this plot can be seen in Figure \ref{fig:ca-feat-cluster}. Looking at the most abundant genes, we can notice that the high consistency is directly caused by the scattered distribution of the cells across multiple islands. HV and MA\_HV sets tend to produce more compact group of points, which might lead to a more relevant biological interpretation. Thus, using the stability assessment as well as the topology, the most suitable option should be the highly variable set having 2500-3000 genes.

\begin{figure}[H]
    \centering
    \makebox[\textwidth][c]{\includegraphics[width=0.6\linewidth]{images/ch3/3_feat_clusters_facet.png}}
    \caption{\label{fig:ca-feat-cluster}\textbf{Assessment of the stability using the cluster distribution} A grid figure where each plot is the cluster representation on a low-dimensional space for each subset of the three feature types. The sparse distribution of points for MA is motivating the high stability from the previous plots. }
\end{figure}

\subsubsection{Stability areas}
This plot complements the previous one and displays the distribution of the ECC score on the UMAP topology, as it can be seen in Figure \ref{fig:ca-feat-stab}. This plot is taking advantage of the useful property of the Element-Centric Similarity score and allows the user to identify the stable group of cells and the areas where the clustering in not consistent.

\begin{figure}[H]
    \centering
    \makebox[\textwidth][c]{\includegraphics[width=0.6\linewidth]{images/ch3/3_feat_stab_facet.png}}
    \caption{\label{fig:ca-feat-stab}\textbf{Assessment of the stability using the representation of the ECC distribution on a low dimensional space.} A grid figure where each plot is the ECC representation on a low-dimensional space for each subset of the three feature types.}
\end{figure}

\subsection{Assessing the graph construction}
Moving on the graph construction step, the most important parameters are the following:
\begin{itemize}
    \item base embedding: PCA or UMAP
    \item the number of neighbours: it is intuitive to say that using more neighbours would not damage or affect the clustering results; the purpose of including this parameter is that its values have an impact over the execution efficiency of the whole pipeline. Having more neighbours would lead to more calculations. Thus, our package allows the user to search for small values of the $k$ parameter that could be stable and relevant for the desired output.
    \item the graph type: unweighted (NN) or weighted (SNN)
\end{itemize}

Our package does not assess the change of the distance metric, as the \verb|RANN| package does not currently have support for changing this parameter.
For this step, \verb|ClustAssess| can output three types of plots.

\subsubsection{Connected components target}
Since we are using community detection methods for graph clustering, it is unlikely (if not impossible) to obtain a cluster that contains two connected components. This type of merging (or moving a node to a cluster from another connected component) does not bring any improvement on the value of the quality function. One exception is related to the singleton clusters, which can be merged inside one single group or can be appended to the community in the closest proximity.

Therefore, the number of connected components acts like a lower bound for the possible number of clusters. The plot discussed in this section illustrates the impact that the number of nearest neighbours has on the number of connected components. The results are obtained by running the clustering pipeline multiple times while changing the seed. Thus, for each number of $k$, a boxplot distribution of the number of connected components will be displayed. PCA and UMAP can be used as base embedding. 

Figure \ref{fig:ca-conn-comp} shows how increasing the number of neighbours monotonically decreases the number of connected components, which is an expected behaviour, as more neighbours leads to more edges and to better connectivity. We notice how PCA requires only four number of nearest neighbour to reach a connected graph, while UMAP converges to a number of four. 

The utility of this plot resides in the ability to determine the appropriate number of neighbours that should be used to obtain at least a desired number of clusters. In our example, if we would want to get a number of five clusters when using UMAP as base embedding, the plot suggest that we should use at least 10 nearest neighbours.

\begin{figure}[H]
    \centering
    \makebox[\textwidth][c]{\includegraphics[width=0.6\linewidth]{images/ch3/3_conn_comp.png}}
    \caption{\label{fig:ca-conn-comp}\textbf{Connected components target plot, obtained on 30 runs} Each boxplot indicates the distribution of the number of connected components. As the number of neighbours increases, the graph connectivity also increases, which leads to fewer disconnected groups. PCA requires a small number of neighbours to reach a connected graph.}
\end{figure}

\subsubsection{Relationship between the number of clusters and number of neighbours}
The following type of plot graphically represents the relationships the number of neighbours and the actual number of clusters. Besides the graph embedding, the graph type can also be changed. The number of clusters is displayed as a boxplot distribution (see explanation at the previous plot). This plot can be used to infer the expected number of clusters on different configuration of these three parameters when the clustering algorithms is used with its default settings.

An example is provided in Figure \ref{fig:ca-nn-k}. We can notice the same decreasing tendency as in the connected components case when using more nearest neighbours. In our case, the graph type does not have a great impact over the number of clusters, but the base embedding does: PCA-based graphs are clustered in less communities.

\begin{figure}[H]
    \centering
    \makebox[\textwidth][c]{\includegraphics[width=0.6\linewidth]{images/ch3/3_nn_k.png}}
    \caption{\label{fig:ca-nn-k}\textbf{The number of clusters - number of neighbours relationship plot, obtained after 30 runs.} Similarly to the previous plot, the boxplots indicate the distribution of the obtained number of clusters. The tendency noticed in the connected components plots is also observed here. PCA tends to obtain fewer clusters than UMAP. }
\end{figure}

\subsubsection{Graph building stability}
This plot evaluates the stability of different configurations of the three parameters involved in the graph construction. Its use-case is to help the user choose the values that lead to robustness at changes of the random seed. Similarly to the Feature stability distribution plot, the stability is assessed using the ECC score. Its distribution is displayed using a boxplot distribution. The boxplots are grouped based on the different configurations.

For example, if Figure \ref{fig:ca-nn-ecc} we have four configurations as a result of combining the PCA and UMAP embedding with NN and SNN graph types. THere are more conclusions to be drawn from this plot:
\begin{itemize}
    \item given that UMAP is more reliant on the initial seed than the PCA, the results obtained from the latter are significantly more consistent.
    \item increasing the number of neighbours leads to the improvement of the stability.
    \item adding weights to the graph might lead to more consistent results.
\end{itemize}
\begin{figure}[H]
    \centering
    \makebox[\textwidth][c]{\includegraphics[width=0.6\linewidth]{images/ch3/3_nn_ecc.png}}
    \caption{\label{fig:ca-nn-ecc}\textbf{The graph building stability plot, obtained after 30 runs.} The stability is measured by calculating the ECC of the partitions obtained across seed changes. The ECC distribution is displayed using boxplots. Each colour is associated with a configuration that contains the following parameters: base embedding, graph type and ECS threshold. UMAP is more affected by the change of random seed, hence lower ECC values than PCA. Increasing the number of neighbours improves the overall consistency. }
\end{figure}

\subsection{Assessing the graph clustering}
Once the dimensionality reduction is performed and the graph is built, the remaining task is to assess the stability of the clustering. \verb|ClustAssess| offers visual tools to evaluate the consistency of the community detection method, as well as of its most relevant parameter, the resolution.

\subsubsection{Clustering method stability}
ClustAssess uses Seurat for the implementation of the Phenograph pipeline, thus the available community detection methods are Louvain \cite{Blondel2008b}, Louvain with multi-level refinement \cite{Rotta2011}, SLM \cite{Waltman2013} and Leiden \cite{Traag2019a}. The following plot illustrates, for a given range for the resolution parameter, the distribution of the ECC. The procedure is slightly different from the previous steps: instead of taking all the partitions obtained after multiple runs with different seeds, we extract the number of clusters that appears the most frequent and we calculate the consistency of the clusterings having that number of groups. Above each boxplot the most common number of clusters will be displayed. This enables comparing the community detection methods not only by stability, but also by the number of clusters they will most probably output.

An example is provided in Figure \ref{fig:ca-clust-dif-boxplot} where the stability of the four clustering algorithms are assessed on the Cuomo dataset. Leiden (with red colour) followed by Louvain (green) are noticeably less consistent than the other two methods, thus the conclusion the user could draw from this plot is that Louvain refined or SLM are more suitable choices to reach robust results.

\begin{figure}[H]
    \centering
    \makebox[\textwidth][c]{\includegraphics[width=0.6\linewidth]{images/ch3/3_clust_dif_boxplot.png}}
    \caption{\label{fig:ca-clust-dif-boxplot}\textbf{Clustering method stability plot, obtained after 30 runs.} Each colour is associated with a different clustering technique. The boxplot illustrate the distribution of the ECC score. Above each boxplot the number of clusters that appears the most frequently is displayed. Leiden and Louvain are less stable than Louvain refined and SLM on some resolution values.}
\end{figure}

\subsubsection{Areas of clustering consistency}
To reinforce the previous plot, \verb|ClustAssess| also offers support for visualising the distribution of the ECC score on an UMAP representation. The plot is faceted, so each column will be associated with one clustering method and each row with a resolution value. This plot can serve as a tool to identify the areas that are unstable, as well to assess whether the instability occurs at all algorithms in the same areas or not. We note that, in this case, the ECC will be calculated across all partitions that are obtained without filtering by a specific number of clusters.

An example is provided in Figure \ref{fig:ca-clust-dif-facet}. It's noticeable how Leiden's behaviour is different from the other three: the main area of instability is located in the upper island. Louvain also has more inconsistent areas when resolution is greater than 0.6. SLM and Louvain refined seem to behave similarly with few exceptions: for resolution 0.5 or 0.7, where the latter has an additional unstable region. The conclusion the user could draw is that SLM has the most robust behaviour, as there are a few number of unstable areas; in addition to that, those areas are unstable in the other algorithms as well.

\begin{figure}[H]
    \centering
    \makebox[\textwidth][c]{\includegraphics[width=0.8\linewidth]{images/ch3/3_clust_dif_facet.png}}
    \caption{\label{fig:ca-clust-dif-facet}\textbf{Areas of clustering consistency plot, obtained after 30 runs}. A facet plot where each row is associated with a resolution value and a column with a clustering method. Each panel presents the ECC score distribution on a low-dimensional space. Leiden and Louvain have the most different behaviours when it comes to obtaining unstable regions. SLM has the most robust results. The areas where it is incosistent are in agreement with the results of the other community detection methods.}
\end{figure}

\subsubsection{Resolution - number of cluster correspondence}
One of the limitation of the community detection method is that there is no direct way to specify the number of desired clusters, like in other clustering algorithms such as k-means. The number of clusters is directly affected by the resolution parameter, but the relationship between these two values is dependent on the graph's structure.

The following plot is meant to illustrate this correspondence between the two parameters. On the X-axis different resolution values provided by the user are displayed, and the Y-axis contains the number of clusters. The correspondence is displayed using a scatter plot. One immediate use-case for it is to identify the suitable resolution values to obtain a desired number of clusters. 

This plot also offers additional information regarding the stability of the resolution parameter - number of clusters pair. Firstly, the point size is proportional to either the frequency of the most common partition or to the Element-Centric Consistency of the list of partitions which are obtained at different combinations of the pairs. The purpose of this information is to help the user evaluate whether the number of clusters obtained on a specific resolution value is robust to seed change or not.

The colour gradient is used to illustrate the frequency of the partitions with a given number of clusters relative to the number of runs on a specific resolution value. The values associated with the colours should sum up to the value of one for each resolution. The purpose of this type of information is to provide statistical assurance on the stability assessments. Brighter colours (close to yellow) indicate a high frequency of that number of clusters, therefore the conclusion drawn from the point size could be trusted. However, the darker shades (close to dark blue) indicate that the conclusion is drawn on a sample too small to provide confidence in the results.

The shape of points is meant to illustrate different configurations that are used for the clustering. In this case, a configuration contains the number of nearest neighbours used in the graph building, the graph type and the clustering algorithm.

An example is provided in Figure \ref{fig:ca-1-kres}. We notice how increasing the resolution leads to partition with more clusters. For example, we can obtain 13 clusters for the configuration \verb|30_SNN_Louvain| for both resolutions 0.5 and 0.6. Also, for the same configuration, there is no resolution value that outputs partitions with 17 clusters. We must note that although there are situations when some numbers of clusters cannot be obtained, a more dense range should be used i.e. with a distance between values lower than 0.1 to draw this type of conclusions.

\begin{figure}[H]
    \centering
    \makebox[\textwidth][c]{
        \includegraphics[width=0.5\linewidth]{images/ch3/3_1_kres_freq.png}
        \includegraphics[width=0.5\linewidth]{images/ch3/3_1_kres_ECC.png}}
    \caption{\label{fig:ca-1-kres}\textbf{Resolution - number of clusters correspondence plot.} The X-axis represents different resolution values and the Y-axis the number of obtained clusters. The colour indicates the frequency of the partitions with a given number of clusters for a fixed resolution value. For each resolution, the values indicated by the colour should sum up to 1. Each point shape is associated with a different configuration of parameters. The point size can illustrate either the frequency of the most common partition (left panel), or the ECC of the list of partitions with $c$ clusters obtained on resolution $\gamma$ (right panel). The point size indicates the robustness of the configuration at the change of random seed (bigger point sizes should be translate to high consistency). The colour gradient helps the user decide whether the stability conclusions are drawn from a statistically relevant sample (brighter shades are associated with relevant samples). The plot can be used to infer the appropriate resolution value for a desired number of clusters. }
\end{figure}

We note some cases of high stability indicated by both ECC and partition frequency, but not backed up by a relevant sample (see the points when resolution is 0.6 and the number of clusters 13). The plots also contains situations where the sample is high enough to trust the conclusion that a specific number of cluster is unstable (see the two configuration that lead to partitions with 17 clusters at resolution 1). The number of 18 clusters seems to be a robust choice for the configurations based on the SNN graph type.

\subsubsection{Stability of the number of clusters}
As can be observed from Figure \ref{fig:ca-1-kres}, multiple values of the resolution parameter can lead to partitions with the same number of clusters. Therefore, the task evolves from identifying stable pairs of resolution - number of clusters values to finding the number of clusters that is stable for any resolution value.

This plot removes the resolution dependency, so the partitions with the same number of communities are kept into the same list, compared to the previous approach, where they were separated by the extra-layer added by resolution. The scatter plot approach is kept, but on the X-axis the number of clusters is displayed, whereas the Y-axis contains the number of unique partitions with the same number of clusters. This is the first information that could be used in order to infer the stability: if there are many unique partitions that could be obtained, it is probable that the number of clusters is unstable.

The colour gradient provides additional information regarding the stability. Currently, there are two options for the information transmitted through the colour gradient: either the frequency of the most common partition, or the ECC of the list of partitions with the same number of clusters. Lighter shades (close to yellow) indicate high robustness, while low shades (close to dark blue) define instability. Both the colour gradient and the number of unique partitions can be used in order to assess the consistency of the number of clusters.

There could be situations when a number of clusters is present in only one partition that appears once. Although it could be considered stable, the data is insufficient to draw a conclusion. To verify the reliability of the previously-drawn conclusions, the point size is used to show the frequency of the partitions having a specific number of clusters. Thus, large points are indicator that the stability inference is made on a sufficient sample of partitions and can be trusted.

Similarly to the previous plot, the shape of points indicate different configurations where the varying parameters are the number of nearest neighbour, the graph type and the community detection algorithm.

An example is presented in Figure \ref{fig:ca-1-knpart}. The complexity of the plot allows to make multiple observations, such as the instability of the 17 as the number of clusters for "30\_NN\_Louvain" and "30\_N\_Louvain.refined". Also, we can notice that it is not redundant to use analyze both cases when the colour gradient represents the frequency of the most common partition (left panel) or the ECC of the list of partitions (right panel). There are cases, as the one when the number of clusters is 15, when the frequency is not convincing, but the high consistency provides the additional information that the other partitions are not significantly different from the most frequent one. We can remark the importance of the point size for 13 number of clusters with "30\_NN\_Louvain.refined" configuration: the stability is ranked high by both indicators, but the size of the point indicates that the sample used to calculate the stability is too small to be trusted. Two suitable choices that can be immediately spotted are 16 and 18, where there is high stability, a small number of different partitions and a reasonable sample size.

\begin{figure}[H]
    \centering
    \makebox[\textwidth][c]{
        \includegraphics[width=0.5\linewidth]{images/ch3/3_1_knpart_freq.png}
        \includegraphics[width=0.5\linewidth]{images/ch3/3_1_knpart_ECC.png}}
    \caption{\label{fig:ca-1-knpart}\textbf{Stability of the number of clusters plot.} The X-axis represents the number of clusters and the Y-axis the number of different partitions with a specific number of clusters that were obtained. The point size indicates the frequency of the partitions with a given number of clusters. Each point shape is associated with a different configuration of parameters. The colour can illustrate either the frequency of the most common partition (left panel), or the ECC of the list of partitions (right panel). The colour, as well as the y value, indicate the robustness of the configuration at the change of random seed (brighter shades of colour and low y value are a sign of stability). The point size helps the user decide whether the stability conclusions are drawn from a statistically relevant sample (big point sizes indicate relevant samples).}
\end{figure}

%\begin{figure}[H]
%    \centering
%    \makebox[\textwidth][c]{
%        \includegraphics[width=0.5\linewidth]{images/ch3/3_99_kres_freq.png}
%        \includegraphics[width=0.5\linewidth]{images/ch3/3_99_kres_ECC.png}
%    }
%    \caption{\label{fig:ca-99-kres}Clustering distribution with default parameters for Monocle (left) and Seurat (right). The title indicates the default random seed that is used.}
%\end{figure}
%\begin{figure}[H]
%    \centering
%    \makebox[\textwidth][c]{
%        \includegraphics[width=0.5\linewidth]{images/ch3/3_99_knpart_freq.png}
%        \includegraphics[width=0.5\linewidth]{images/ch3/3_99_knpart_ECC.png}}
%    \caption{\label{fig:ca-99-knpart}Clustering distribution with default parameters for Monocle (left) and Seurat (right). The title indicates the default random seed that is used.}
%\end{figure}

    \chapter{Experiments and results}

\section{ECS optimization results}


\section{Stability pipeline performance}
    
    \chapter*{Conclusions and Future Work} 
\addcontentsline{toc}{chapter}{Conclusions and Future Work}

Clustering comparison metrics are important for benchmarking and evaluating the output of a clustering method when a ground truth is provided. These scores can also be used to compare different clustering methods or different configurations i.e. different values for the parameters of the same algorithm. Element-Centric Similarity (ECS) is a score that is not biased on the size, shape or the number of clusters and which provides information about the global and local agreement between two partitions. The \verb|ClustAssess| package provides the first R implementation of this score which also performs the calculation in an scalable, both space and time optimized manner.

We also emphasized the importance of the parameter configuration on the clustering output, but also the undesired impact that factors such as random seed can have on the final results. The solution we proposed in \verb|ClustAssess| was to provide a stability pipeline, which follows the steps from the PhenoGraph algorithm. The purpose of the pipeline is to provide visual assessments of the stability and robustness of different configuration at the change of random seed and to assist the user into choosing a configuration that leads to consistent, not seed reliant results.

As future work we are looking to furtherly optimize the calculation of the ECS score based on the final form as it was presented in Remark \ref{remark:ecs-constant}. At the moment only the case of disjoint partitions was optimized, so the calculation of the ECS score between overlapping or hierarhical partitions is also something to be improved. 

The introduction of the ECS threshold term and of the evaluation of close similarity between partitions could lead to further improvements in the time execution. The issue is that a specific value of the ECS value changes its interpretation when the size of the datasets changes. A ECS value of 0.99 could be interpreted as only a few points labeled differently for small datasets or entire region differences for larger ones. We intend in finding a correspondence between the ECS value and the possible number of points that changed their cluster label.

Finally, we are constantly planning on improving the quality and the usefulness of the plots involved in the stability pipeline.

   % \newpage
%
   % Title: ClustAssess: Tools for Assessing Clustering
%
   % Student name: Munteanu Andi
   % 
   % Coordinator: Conf dr. Liviu Ciortuz \\
%
   % The thesis is based on the ClustAssess paper \cite{clustassess}, where I contributed as an author and developer of the R package. \\
%
   % 
   % Clustering is an unsupervised method that is used to classify and label points based on different similarity metrics. Comparing to the supervised classifiers, clustering algorithms do not require training and using a model and are most suitable when the label of the points are not priorly known and are inferred based on the features that describe them.
%
   % Depending on the approach, clustering methods can be divided in multiple categories, such as centroid-based (k-means), density-based (DBSCAN), hierarchical, distribution-based (EM). One approach that gained popularity in the last decades are the graph-based clustering. Some of its advantages is providing flexibility when it comes to the cluster shapes and sizes, or the scalability.
%
   % Our thesis focus is set on the community detection techniques, that is graph-based clustering approach that rely on optimizing an objective function (also known in literature as quality function or quality metric, as it tries to evaluate the quality of the clustering by encouraging high density of edges inside clusters and as few intrer-cluster links as possible). This method is intensively used, as it manages to obtain close to optimal results in an efficient time. 
%
   % Given that many datasets are provided as points displayed on a high-dimensional space, a methodology that enables the use of community detection method on this data is required. Our reference is the PhenoGraph algorithm presented by Levine et. al \cite{Levine2015}, that establishes a pipeline that firstly applies a dimensionality reduction technique (a linear one such as PCA or non-linear one such as UMAP), then generates a graph using the Nearest Neighbour algorithm. The resulting graph is then used as input for the community detection algorithm.
%
   % The PhenoGraph pipeline has been used frequently in several domains. One of them is the downstream analysis of biological data, where the goal is to cluster the cells in multiple groups that will be used to infer some biological conclusions. The input data is usually provided as a matrix that has cells sampled from different donors at different timepoints on the rows. The columns represent the genes that can be expressed in the cells. Given that the human genome contains approximately 30.000 genes, the input space is a high dimensional one, which makes the PhenoGraph pipeline a suitable choice for processing the data. Also, using graphs to represent data is a more appropiate way to describe the cell-cell interaction.
%
   % The current state-of the art of processing the biological data is thus using the PhenoGraph pipeline: for dimensionality reduction, approximate PCA using the Lanczos bidiagonalization method \cite{Baglama2016IRLBAFP} or UMAP \cite{mcinnes2018uniform}; for graph building, the Nearest Neighbour algorithm \cite{Xu2015}; for community detection, Louvain \cite{Blondel2008b}, Louvain with multilevel refinement \cite{Rotta2011}, Smart Local Moving Algorithm \cite{Waltman2013} and Leiden \cite{Traag2019a}. 
%
   % Currently, some of the most popular tools used for analyzing single-cell data are Seurat \cite{Hao2021}, Monocle \cite{Cao2019} and SCANPY \cite{Wolf2018}. Our thesis makes a comparison between Seurat and Monocle regarding the implementation of the PhenoGraph pipeline. This was motivated by the significant differences between the results of the two packages and the subsequent divergent biological interpretation of the obtained partitions. The question we wanted to answer is whether this differences are caused by computational or biological factors (such as sequencing depth or how the data was pre-processed).
%
   % In our thesis we showcase how the divergence was caused by using parameters values that do not match. The conclusion we draw is that tuning the algorithm's parameters is essential in obtaining reproducable results. Given the stochastic nature of the algorithms that are involved in the pipeline, we also noticed how changing the random seed value is a direct factor that affects the clustering output.
%
   % The instability caused by random seed was previously identified in several papers that pursued the algorithm modification in order to achieve stability. Such example is kmeans++ \cite{kmeanspp}, where the authors replaced the random initialization of the centroids with assigning probabilities of selection based on the distance to the existing center points. Another example is provided in the clust-perturb algorithm proposed by Stacey et al. \cite{STACEY2021}, where the robustness is evaluated by introducing random noise in the graph.
%
   % Our work is focused on providing a pipeline that follows the algorithms involved in the PhenoGraph. The package that we developed its purposed to provide informative plots that would give the user insight about how different parameters impact the number of clusters and the partitioning. Another purpose that we try to achieve is to evaluate and provide insight about parameters configuration that are robust to the change of seeds. The robustness is determined by using Element Centric Similarity (ECS) \cite{Gates2019}, a measurement that determines how similar are two clustering of the same data. 




    \bibliographystyle{unsrt}
    \bibliography{chapters/bibliography}{}
    
\end{document}
